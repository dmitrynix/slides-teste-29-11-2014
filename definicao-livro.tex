\begin{frame}
  % Fiquei empolgado quando vi um livro de "engenharia de software" falando sobre
  % teste de software, ainda mais sendo um livro antigo.
  \begin{center}
    \Huge Definição de um livro
  \end{center}
\end{frame}

\begin{frame}{\subsecname}
  \blockquote[{\cite[Pressman]{pressman_engenharia_1995}}]{
    [\ldots] \emph{Não é incomum} que uma organização de software gaste 40\%
    do esforço de projeto total em teste\ldots

    \let\thefootnote\relax\footnote{Grifo meu}
  }
  % "Não é incomum", gostei de você
\end{frame}

\begin{frame}{\subsecname}
  \blockquote[{\cite[Pressman]{pressman_engenharia_1995}}]{
    [\ldots] Surge a fase de testes. O engenheiro cria uma série de casos de testes
    que têm a intenção de "demolir" o software que ele construiu.
  }
  % WTF? "Fases de testes"?
\end{frame}

\begin{frame}{\subsecname}
  \begin{center}
    \includegraphics[width=5cm]{images/mother-of-god}
  \end{center}
  % No, no, no, no, no
\end{frame}

\begin{frame}{\subsecname}
  % Já imaginou se as motadoras de carro testassem carros assim?
  \begin{center}
    \includegraphics[width=5cm]{images/bill_gm}
  \end{center}
\end{frame}

\begin{frame}{\subsecname}
  \begin{center}
    \includegraphics[width=5cm]{images/rebeldia}
  \end{center}
\end{frame}

\begin{frame}{\subsecname}
  \blockquote[{\cite[Pressman]{pressman_engenharia_1995}}]{
    18.1.1 Objetivo das atividades de testes

    [\ldots]
    \begin{enumerate}
      \item A atividade de teste é o processo de \emph{executar um programa com
          a intenção de descobrir um erro ainda não descoberto}.
      \item Um \emph{bom caso de teste} é aquele que \emph{tem uma elevada
        probabilidade de revelar um erro ainda não descoberto}.
      \item Um teste \emph{bem-sucedido} é aquele que \emph{revela um erro
          ainda não descoberto}.
    \end{enumerate}

    \let\thefootnote\relax\footnote{Grifo meu}
  }
\end{frame}

\begin{frame}{\subsecname}
  \blockquote[{\cite[Pressman]{pressman_engenharia_1995}}]{
    Se a atividade de teste for conduzida com sucesso \ldots, ela descobrirá
    erros no software. Como um \emph{benefício secundário, a atividade de
    teste demonstra que as funções de software aparentemente estão trabalhando de
    acordo com as especificações}, que os requisitos de desempenho aparentemente
    está cumprido.

    \let\thefootnote\relax\footnote{Grifo meu}
  }
\end{frame}

\begin{frame}{\subsecname}
  \blockquote[{\cite[Pressman]{pressman_engenharia_1995}}]{
    18.7 Ferramentas de teste automatizadas
  }
  % Falou alguns conceitos legais de teste, mas outros completamente nadazavê
\end{frame}

\begin{frame}{\subsecname}
  \blockquote[{\cite[Pressman]{pressman_engenharia_1995}}]{
    18.8 Resumo

    O objetivo principal do projeto de casos de testes é derivar um conjunto de
    testes, que tenha uma alta probabilidade de revelar defeitos no software.
  }
\end{frame}

% O livro até que tem conceitos bons, mas bate na tecla de que teste deve ser
% feito depois que o programa estiver pronto e que tem que descobrir erros =/
