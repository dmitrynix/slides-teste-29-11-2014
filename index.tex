\documentclass[]{beamer}

\usepackage[utf8]{inputenc}
\usepackage[portuges]{babel}
\usepackage[T1]{fontenc}
\usepackage{hyperref}
\usepackage[]{csquotes}

\setbeamertemplate{footline}[page number]
\setbeamercovered{transparent}
\beamertemplatenavigationsymbolsempty

\ifdefined\BLACKBG
  \setbeamercolor{block body}{bg=normal text.bg!90!black}
  \setbeamercolor{block body example}{bg=normal text.bg!90!black}
  \setbeamercolor{block title alerted}{use={normal text,alerted text},fg=alerted text.fg!75!normal text.fg,bg=normal text.bg!75!white}
  \setbeamercolor{block title}{bg=black}
  \setbeamercolor{block title example}{use={normal text,example text},fg=example text.fg!75!normal text.fg,bg=normal text.bg!75!white}
  \setbeamercolor{fine separation line}{}
  \setbeamercolor{frametitle}{fg=white}
  \setbeamercolor{item projected}{fg=white}
  \setbeamercolor{normal text}{bg=black,fg=white}
  \setbeamercolor{separation line}{}
  \setbeamercolor{structure}{bg=black,fg=white}
  \setbeamercolor{title}{fg=white}
  \setbeamercolor{titlelike}{fg=white}
\else
  \setbeamercolor{block body}{bg=normal text.bg!90!white}
  \setbeamercolor{block body example}{bg=normal text.bg!90!white}
  \setbeamercolor{block title alerted}{use={normal text,alerted text},fg=alerted text.fg!75!normal text.fg,bg=normal text.bg!75!black}
  \setbeamercolor{block title}{bg=white}
  \setbeamercolor{block title example}{use={normal text,example text},fg=example text.fg!75!normal text.fg,bg=normal text.bg!75!black}
  \setbeamercolor{fine separation line}{}
  \setbeamercolor{frametitle}{fg=black}
  \setbeamercolor{item projected}{fg=black}
  \setbeamercolor{normal text}{bg=white,fg=black}
  \setbeamercolor{separation line}{}
  \setbeamercolor{structure}{bg=white,fg=black}
  \setbeamercolor{title}{fg=black}
  \setbeamercolor{titlelike}{fg=black}
\fi

\newcommand{\invertbgcolor}{
  \ifdefined\BLACKBG
    %\setbeamercolor{normal text}{bg=white,fg=black}
  \fi
}

\newcommand{\backbgcolor}{
  \ifdefined\BLACKBG
    %\setbeamercolor{normal text}{bg=white,fg=black}
  \fi
}


\author{Dmitry "Nix" Rocha}
\title{Aprendendo Python com testes}
\institute{ /(guru|pug-)pi/i }

\begin{document}
  \begin{frame}{}
    \titlepage
  \end{frame}

  \begin{frame}{su - mario}
    \tableofcontents
  \end{frame}

  \section{O que é teste automatizado de software?}

  \subsection*{Definição}
  \begin{frame}
    \begin{center}
      \Huge Definição
    \end{center}
  \end{frame}

  \begin{frame}{}
    \begin{block}{Teste}
      É algo que testo =/
    \end{block}
    \pause

    \begin{block}{Automatizado}
      É algo que automatizo :P
    \end{block}
    \pause

    \begin{block}{Software}
      Conjunto de instruções.
    \end{block}
    \pause
  \end{frame}

  \setbeamercolor{normal text}{bg=white,fg=black}
  \begin{frame}{}
    \begin{center}
      \includegraphics[width=5cm]{images/pokerface}
    \end{center}
  \end{frame}

  % back to black-on-white
  \setbeamercolor{normal text}{bg=black,fg=white}
  \begin{frame}{Outra Definição}
\blockquote[{\cite[Wikipedia]{_test_2014}}]{
In software testing, test automation is the use of special software (separate
from the software being tested) to control the execution of tests and the
comparison of actual outcomes with predicted outcomes. Test automation can
automate some repetitive but necessary tasks in a formalized testing process
already in place, or add additional testing that would be difficult to perform
manually.
}
  \end{frame}

  \section{Exemplos e casos de uso com Python}

  \begin{frame}{fibonacci}
    Sequência de Fibonacci :)
  \end{frame}

  \bibliographystyle{plain}
  \bibliography{Items}
\end{document}
