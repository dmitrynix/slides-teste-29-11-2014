\documentclass[]{beamer}
\mode<presentation>

\usepackage[utf8]{inputenc}
\usepackage[portuges]{babel}
\usepackage[T1]{fontenc}
%\usepackage{hyperref}
\usepackage[]{csquotes}

\setbeamertemplate{footline}[page number]
\setbeamercovered{transparent}
\beamertemplatenavigationsymbolsempty

%\def\BLACKBG{true}
\ifdefined\BLACKBG
  \setbeamercolor{block body}{bg=normal text.bg!90!black}
  \setbeamercolor{block body example}{bg=normal text.bg!90!black}
  \setbeamercolor{block title alerted}{use={normal text,alerted text},fg=alerted text.fg!75!normal text.fg,bg=normal text.bg!75!white}
  \setbeamercolor{block title}{bg=black}
  \setbeamercolor{block title example}{use={normal text,example text},fg=example text.fg!75!normal text.fg,bg=normal text.bg!75!white}
  \setbeamercolor{fine separation line}{}
  \setbeamercolor{frametitle}{fg=white}
  \setbeamercolor{item projected}{fg=white}
  \setbeamercolor{normal text}{bg=black,fg=white}
  \setbeamercolor{separation line}{}
  \setbeamercolor{structure}{bg=black,fg=white}
  \setbeamercolor{title}{fg=white}
  \setbeamercolor{titlelike}{fg=white}
\else
  \setbeamercolor{block body}{bg=normal text.bg!90!white}
  \setbeamercolor{block body example}{bg=normal text.bg!90!white}
  \setbeamercolor{block title alerted}{use={normal text,alerted text},fg=alerted text.fg!75!normal text.fg,bg=normal text.bg!75!black}
  \setbeamercolor{block title}{bg=white}
  \setbeamercolor{block title example}{use={normal text,example text},fg=example text.fg!75!normal text.fg,bg=normal text.bg!75!black}
  \setbeamercolor{fine separation line}{}
  \setbeamercolor{frametitle}{fg=black}
  \setbeamercolor{item projected}{fg=black}
  \setbeamercolor{normal text}{bg=white,fg=black}
  \setbeamercolor{separation line}{}
  \setbeamercolor{structure}{bg=white,fg=black}
  \setbeamercolor{title}{fg=black}
  \setbeamercolor{titlelike}{fg=black}
\fi

\newcommand{\invertbgcolor}{
  \ifdefined\BLACKBG
    %\setbeamercolor{normal text}{bg=white,fg=black}
  \fi
}

\newcommand{\backbgcolor}{
  \ifdefined\BLACKBG
    %\setbeamercolor{normal text}{bg=white,fg=black}
  \fi
}


\author[Dmitry]{Dmitry Nix \\ \texttt{@dmitrynix}}

\title{Testes, testes everywhere}
\institute{ /(guru|pug-)pi/i }
\date{29 de Novembro de 2014}

\begin{document}
  \begin{frame}
    \titlepage
    % Sobre mim:
    % - Informática;
    % - Matemática;
    % - GURUPI/PUG-PI;
    % - scoola;
  \end{frame}

  \begin{frame}{su - mario}
    \tableofcontents
    % Dividi a talk em duas:
    %  1. Fundamentação teórica
    %  2. Prática com Ruby, também pode ser chamada de boas práticas
  \end{frame}

  \section{O que é teste automatizado de software?}

  \begin{frame}
    \begin{center}
      \Huge Definições
    \end{center}
  \end{frame}

  %
  %
  %  Definição de Livro
  %
  %

  \subsection*{Definição de Livro}\label{def1}
  \begin{frame}
    % Fiquei empolgado quando vi um livro de "engenharia de software" falando sobre
    % teste de software, ainda mais sendo um livro antigo.
    \begin{center}
      \Huge Definição de um livro
    \end{center}
  \end{frame}

  \begin{frame}{\subsecname}
    \blockquote[{\cite[Pressman]{pressman_engenharia_1995}}]{
      [\ldots] \emph{Não é incomum} que uma organização de software gaste 40\%
      do esforço de projeto total em teste\ldots

      \let\thefootnote\relax\footnote{Grifo meu}
    }
    % "Não é incomum", gostei de você
  \end{frame}

  \begin{frame}{\subsecname}
    \blockquote[{\cite[Pressman]{pressman_engenharia_1995}}]{
      [\ldots] Surge a fase de testes. O engenheiro cria uma série de casos de testes
      que têm a intenção de "demolir" o software que ele construiu.
    }
    % WTF? "Fases de testes"?
  \end{frame}

  \begin{frame}{\subsecname}
    \begin{center}
      \includegraphics[width=5cm]{images/mother-of-god}
    \end{center}
    % No, no, no, no, no
  \end{frame}

  \begin{frame}{\subsecname}
    % Já imaginou se as motadoras de carro testassem carros assim?
    \begin{center}
      \includegraphics[width=5cm]{images/bill_gm}
    \end{center}
  \end{frame}

  \begin{frame}{\subsecname}
    \begin{center}
      \includegraphics[width=5cm]{images/rebeldia}
    \end{center}
  \end{frame}

  \begin{frame}{\subsecname}
    \blockquote[{\cite[Pressman]{pressman_engenharia_1995}}]{
      18.1.1 Objetivo das atividades de testes

      [\ldots]
      \begin{enumerate}
        \item A atividade de teste é o processo de \emph{executar um programa com
            a intenção de descobrir um erro ainda não descoberto}.
        \item Um \emph{bom caso de teste} é aquele que \emph{tem uma elevada
          probabilidade de revelar um erro ainda não descoberto}.
        \item Um teste \emph{bem-sucedido} é aquele que \emph{revela um erro
            ainda não descoberto}.
      \end{enumerate}

      \let\thefootnote\relax\footnote{Grifo meu}
    }
  \end{frame}

  \begin{frame}{\subsecname}
    \blockquote[{\cite[Pressman]{pressman_engenharia_1995}}]{
      Se a atividade de teste for conduzida com sucesso \ldots, ela descobrirá
      erros no software. Como um \emph{benefício secundário, a atividade de
      teste demonstra que as funções de software aparentemente estão trabalhando de
      acordo com as especificações}, que os requisitos de desempenho aparentemente
      está cumprido.

      \let\thefootnote\relax\footnote{Grifo meu}
    }
  \end{frame}

  \begin{frame}{\subsecname}
    \blockquote[{\cite[Pressman]{pressman_engenharia_1995}}]{
      18.7 Ferramentas de teste automatizadas
    }
    % Falou alguns conceitos legais de teste, mas outros completamente nadazavê
  \end{frame}

  \begin{frame}{\subsecname}
    \blockquote[{\cite[Pressman]{pressman_engenharia_1995}}]{
      18.8 Resumo

      O objetivo principal do projeto de casos de testes é derivar um conjunto de
      testes, que tenha uma alta probabilidade de revelar defeitos no software.
    }
  \end{frame}

  % O livro até que tem conceitos bons, mas bate na tecla de que teste deve ser
  % feito depois que o programa estiver pronto e que tem que descobrir erros =/

  %
  %
  %  Definição da wikipedia
  %
  %

  \subsection*{Definição da Wikipedia}
  \begin{frame}
    \begin{center}
      \Huge Definição da Wikipedia
    \end{center}
  \end{frame}

  \begin{frame}{\subsecname}
    \blockquote[{\cite[Wikipedia]{_test_2014}}]{
      In software testing, test automation is the use of special software
      (separate from the software being tested) to control the execution of
      tests and the comparison of actual outcomes with predicted outcomes. Test
      automation can automate some repetitive but necessary tasks in a
      formalized testing process already in place, or add additional testing
      that would be difficult to perform manually.
    }
  \end{frame}

  \begin{frame}{\subsecname}
    \blockquote[{\cite[Wikipedia]{_test_2014}}]{
      Em teste de software, teste automatizado é um uso especial do software
      (separado do software que será testado) para controlar a execução do testes
      e comparar a saída com a saída que se espera. Teste automatizado pode
      efetuar algumas tarefas repetitivas, [\ldots]
    }
  \end{frame}

  \begin{frame}{}
    \begin{center}
      \includegraphics[width=5cm]{images/like-a-sir}
    \end{center}
  \end{frame}

  %
  %
  % Minha definição
  %
  %

  \subsection*{Minha definição}

  \begin{frame}
    \begin{center}
      \Huge Minha definição
    \end{center}
  \end{frame}

  \begin{frame}[label=minha-definicao]{\subsecname}
    \begin{block}{Teste}
      É algo que testo para ver se está funcionando =/
    \end{block}
    \pause

    \begin{block}{Automatizado}
      É algo que automatizo :P
    \end{block}
    \pause

    \begin{block}{Software}
      Conjunto de instruções.
    \end{block}
  \end{frame}

  \begin{frame}{\subsecname}
    \begin{center}
      \includegraphics[width=5cm]{images/poker-face}
    \end{center}
  \end{frame}

  \againframe{minha-definicao}

  \backbgcolor
  \section{Exemplos e casos de uso com Ruby}

  \begin{frame}{fibonacci}
    Sequência de Fibonacci :)
  \end{frame}

  \bibliographystyle{plain}
  \bibliography{Items}

  \begin{frame}{}
    \begin{center}
      \includegraphics[width=5cm]{images/isso-e-tudo}
    \end{center}
    % http://lokotopia.com.br/wp-content/uploads/2013/06/013-stefan-asafti-marcas-cartoon-isso-e-tudo.jpg
  \end{frame}
\end{document}
