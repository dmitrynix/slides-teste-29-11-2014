\documentclass{beamer}

\usepackage[utf8]{inputenc}
\usepackage[portuges]{babel}
\usepackage[T1]{fontenc}

\usepackage[]{csquotes}

\author{Dmitry "Nix" Rocha}
\title{Aprendendo Python com testes}

% numeração dos slides por section
\setbeamertemplate{footline}[frame number]
% transparencia nos overlays
\setbeamercovered{transparent}
% Desativando os botoes de navegacao
\beamertemplatenavigationsymbolsempty

\usepackage{hyperref}
\begin{document}
  \begin{frame}{}
    \titlepage
  \end{frame}

  \begin{frame}{su - mario}
    \tableofcontents
  \end{frame}

  \section{O que é teste automatizado de software?}

  \begin{frame}{Definição}
    \begin{block}{Teste}<1->
      É algo que testo =/
    \end{block}

    \begin{block}{Automatizado}<2->
      É algo que automatizo :P
    \end{block}

    \begin{block}{Software}<3->
      Conjunto de instruções.
    \end{block}
  \end{frame}

  \begin{frame}{Chuck Norris aprova}
    \pgfdeclareimage[width=5cm]{approved}{approved}
    \pgfuseimage{approved}<1>
  \end{frame}

  \begin{frame}{Outra Definição}
\blockquote[{\cite[Wikipedia]{_test_2014}}]{
In software testing, test automation is the use of special software (separate
from the software being tested) to control the execution of tests and the
comparison of actual outcomes with predicted outcomes. Test automation can
automate some repetitive but necessary tasks in a formalized testing process
already in place, or add additional testing that would be difficult to perform
manually.
}
  \end{frame}

  \section{Exemplos e casos de uso com Python}

  \begin{frame}{fibonacci}
    Sequência de Fibonacci :)
  \end{frame}

  \bibliographystyle{plain}
  \bibliography{Items}
\end{document}
